\documentclass[a4paper]{article}
\usepackage{geometry}
\geometry{a4paper,left=25mm,right=25mm, top=30mm, bottom=30mm}
\usepackage{fancyhdr}
\usepackage{graphicx}
\usepackage[utf8]{inputenc}
\usepackage{lastpage}
\usepackage{amsmath}
\usepackage{float}
\usepackage[section]{placeins}
%\usepackage[]{natbib}
\usepackage{listings}
\lstset{language=[Objective]C}
\usepackage{color}
\bibliographystyle{unsrt}
\newcommand{\HRule}{\rule{\linewidth}{0.2mm}}


%-------definitions-----
\newcommand{\Author}{Peter Weilnböck}
\newcommand{\Title}{Numerical Study of astrophysical solutions in modified theory of gravity in the Palatini formalism}
\newcommand{\Date}{17.8.2016}
%--------------------------

\normalsize
\title{\Title}
\author{\Author}
\date{\Date}
\pagestyle{fancy}
%%% Kopfzeile linker Bereich
%      gerade Seite   ungerade Seite
\lhead[ \leftmark   ]{\scriptsize \Date}
%%% Kopfzeile mittlerer Bereich
%      gerade Seite   ungerade Seite
\chead[]{\Title}
%%% Kopfzeile linker Bereich
%      gerade Seite             ungerade Seite
\rhead[\rightmark ]{\scriptsize Weilnböck Peter}

%%% Fußzeile linker Bereich
%      gerade Seite   ungerade Seite
\lfoot[]{}
%%% Fußzeile mittlerer Bereich
% bleibt in diesem Beispiel leer
\cfoot[]{}
%%% Fußzeile rechter Bereich
\rfoot[]{\thepage / \pageref{LastPage}}

\begin{document}

\begin{titlepage}
\begin{center} 
%\textsc{\large Laborübungen Fortgeschrittene Experimentiertechniken}\\%
\textsc{\large Karl-Franzens-Universität Graz}\\[3.0cm]

\textsc{\LARGE Master Thesis}\\[1.0cm]
\HRule \\[0.5cm]
{ \Huge \bfseries \Title}\\[0.2cm]
\HRule \\[1.5cm]
\emph{Author:}
Peter Weilnböck\\[0.5cm]
\emph{Adviser: }Dr.rer.nat. Helios Sanchis Alepuz, Univ.-Prof. Dr.rer.nat. Reinhard Alkofer

\vfill
{\LARGE Date: \Date}
\end{center}
\end{titlepage}

\pagebreak

\tableofcontents
\pagebreak

\section*{Abstract}
$\Psi=e^{i \overrightarrow{k} \overrightarrow{r}} u\left( \overrightarrow{r }\right)$

$T_{\overrightarrow{R}} f(\overrightarrow{r})=f(\overrightarrow{r}+\overrightarrow{R} $

$H |\Psi> = E |\Psi> $

$T_{\overrightarrow{R}}\left[ H(\overrightarrow{r})\Psi(\overrightarrow{r})\right] = H(\overrightarrow{r} + \overrightarrow{R})\Psi(\overrightarrow{r} + \overrightarrow{R})$

$H(\overrightarrow{r} + \overrightarrow{R})= H(\overrightarrow{r})$

$=H(\overrightarrow{r})T_{\overrightarrow{R}}\Psi(\overrightarrow{r})=E T_{\overrightarrow{R}}\Psi(\overrightarrow{r})$

$\left[H , T\right] = 0$

$T_{\overrightarrow{R}}\Psi(\overrightarrow{r}) =  \Psi(\overrightarrow{r} + \overrightarrow{R}) \overset{!}{=} c(\overrightarrow{R}) \Psi(\overrightarrow{r})$

Vom endlichen ins unendliche:
$\sum\limits^{k=N}_{k=1}(1)=N $

$\sum\limits_k \longrightarrow c \int\limits_{-\pi/a}^{\pi/a} \emph{d}k = \dfrac{2 \pi}{a} c \overset{!}{=} N$

$\Rightarrow c = \dfrac{N a}{2 \pi} = \dfrac{L}{2\pi} $ 

$\Rightarrow \sum\limits_k \longrightarrow \dfrac{L}{2\pi} \int\limits_{-\pi/a}^{\pi/a} \emph{d}k$
 
$\Rightarrow \sum\limits_k \longrightarrow \dfrac{V}{\left(2\pi\right)^{3}} \int\limits_{-\pi/a}^{\pi/a} \emph{d}^3k$
 
\section{Ising Model}


\nocite{*}
\bibliography{bibfile}

\end{document}